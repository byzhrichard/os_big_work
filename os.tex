\documentclass[UTF8]{ctexart}
\usepackage{geometry}
\geometry{a4paper, left=2.5cm, right=2.5cm, top=2.5cm, bottom=2.5cm}
\usepackage{amsmath}
\usepackage{amssymb}
\usepackage{multirow}
\usepackage{tabularx}
\usepackage{booktabs}
\usepackage{float}
\usepackage{eso-pic}
\usepackage{graphicx} % 用于插入图片
\usepackage{tikz}     % 用于精确定位
\usepackage{fancyhdr} % 添加这个包来控制页眉页脚
\usepackage{listings}
\usepackage{xcolor} % 颜色支持
\usepackage{minted}        % 代码高亮
\usemintedstyle{colorful}  % 代码风格(friendly、colorful、monokai 等)

\usepackage[hidelinks]{hyperref}


% 设置页眉页脚
\pagestyle{fancy}
\fancyhf{} % 清除所有页眉页脚
\renewcommand{\headrulewidth}{0pt} % 去掉页眉的横线
\fancyfoot[C]{\thepage} % 页脚居中显示页码

% 设置章节标题格式
\title{
    \Huge \textbf{西安电子科技大学} \\
    \vspace{0.5cm} 
    ——\LARGE 操作系统~大作业
}
\author{}
\date{}


\begin{document} %%%%%%%%%%%%%%%%%%%%%%%%%

\AddToShipoutPicture*{
    \AtPageUpperLeft{
        \begin{tikzpicture}[remember picture, overlay]
            \node[anchor=north west, inner sep=0pt] at (1cm,-1cm) {
                \includegraphics[width=3cm]{src/xd_local.png}
            };
        \end{tikzpicture}
    }
}

\maketitle

\begin{center}
    \Huge 实验名称: ~ \textbf{容器思想及Docker使用}
    \\
    \Large ———— ~ 基于Docker的Bert落地
\end{center}

\vspace{1cm}

% 添加个人信息部分
\begin{center}
    \Large
    \begin{tabular}{ll}
        学院: & \underline{人工智能}\\
        & \\
        行政班: & \underline{2320032}\\
        & \\
        教学班: & \underline{AI202105 | 02班}\\
        & \\
        姓名: & \underline{黄煜垒, 王小宁}\\
        & \\
        学号: & \underline{23009201388, 23009200506}\\
        & \\
        日期: & \underline{2025年10月}\\
    \end{tabular}
\end{center}

\pagenumbering{roman}
\newpage %%%%%%%%%%%%%%%%%%%%%%%%%

\tableofcontents % 目录

\newpage %%%%%%%%%%%%%%%%%%%%%%%%%
\pagenumbering{arabic}

%%%%%%%%%%%%%%%%%%%%%%%%%%%%%%%%%%%%%%%%%%%%%%%%%%%%%%%%%%%%
\section{Docker介绍} % 小宁来写,小黄来补充

%%%%%%%%%%%%%%%%%%%%%%%%%%%%%%%%%%%%%%%%%%%%%%%%%%%%%%%%%%%%
\subsection{为什么要Docker} % 小宁来写,小黄来补充

%%%%%%%%%%%%%%%%%%%%%%%%%%%%%%%%%%%%%%%%%%%%%%%%%%%%%%%%%%%%
\subsection{Docker基本命令} % 小宁来写,小黄来补充

%%%%%%%%%%%%%%%%%%%%%%%%%%%%%%%%%%%%%%%%%%%%%%%%%%%%%%%%%%%%
\section{Bert介绍} % 小宁来写,小黄来补充

%%%%%%%%%%%%%%%%%%%%%%%%%%%%%%%%%%%%%%%%%%%%%%%%%%%%%%%%%%%%
\subsection{简述Transformer} % 小宁来写,小黄来补充

%%%%%%%%%%%%%%%%%%%%%%%%%%%%%%%%%%%%%%%%%%%%%%%%%%%%%%%%%%%%
\subsection{简述Bert} % 小宁来写,小黄来补充

%%%%%%%%%%%%%%%%%%%%%%%%%%%%%%%%%%%%%%%%%%%%%%%%%%%%%%%%%%%%
\section{实验环境搭建} % 小宁来写,小黄来补充

%%%%%%%%%%%%%%%%%%%%%%%%%%%%%%%%%%%%%%%%%%%%%%%%%%%%%%%%%%%%
\subsection{操作系统} % 小宁来写,小黄来补充

%%%%%%%%%%%%%%%%%%%%%%%%%%%%%%%%%%%%%%%%%%%%%%%%%%%%%%%%%%%%
\subsection{构建Docker} % 小宁来写,小黄来补充

%%%%%%%%%%%%%%%%%%%%%%%%%%%%%%%%%%%%%%%%%%%%%%%%%%%%%%%%%%%%
\section{实验步骤}

%%%%%%%%%%%%%%%%%%%%%%%%%%%%%%%%%%%%%%%%%%%%%%%%%%%%%%%%%%%%
\subsection{启动容器(cuda)}

%%%%%%%%%%%%%%%%%%%%%%%%%%%%%%%%%%%%%%%%%%%%%%%%%%%%%%%%%%%%
\subsection{启动前端}

%%%%%%%%%%%%%%%%%%%%%%%%%%%%%%%%%%%%%%%%%%%%%%%%%%%%%%%%%%%%
\subsection{启动后端}

%%%%%%%%%%%%%%%%%%%%%%%%%%%%%%%%%%%%%%%%%%%%%%%%%%%%%%%%%%%%
\section{实机展示}

%%%%%%%%%%%%%%%%%%%%%%%%%%%%%%%%%%%%%%%%%%%%%%%%%%%%%%%%%%%%
\section{总结}


\end{document} %%%%%%%%%%%%%%%%%%%%%%%%%