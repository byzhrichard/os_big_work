\documentclass[UTF8]{ctexart}
\usepackage{geometry}
\geometry{a4paper, left=2.5cm, right=2.5cm, top=2.5cm, bottom=2.5cm}
\usepackage{amsmath}
\usepackage{amssymb}
\usepackage{multirow}
\usepackage{tabularx}
\usepackage{booktabs}

\usepackage{enumitem}

\usepackage{float}
\usepackage{eso-pic}
\usepackage{graphicx} % 用于插入图片
\usepackage{tikz}     % 用于精确定位
\usepackage{fancyhdr} % 添加这个包来控制页眉页脚
\usepackage{listings}
\usepackage{xcolor} % 颜色支持
\usepackage{minted}        % 代码高亮
\usemintedstyle{colorful}  % 代码风格(friendly、colorful、monokai 等)

\usepackage[most]{tcolorbox}  % 必须!启用 tcolorbox 的高级功能
\usepackage{xcolor}           % 颜色支持
\usepackage{enumitem}         % 控制 enumerate 缩进与标签格式
\usepackage{minted}           % 代码高亮

\usepackage[hidelinks]{hyperref}

% 设置页眉页脚
\pagestyle{fancy}
\fancyhf{} % 清除所有页眉页脚
\renewcommand{\headrulewidth}{0pt} % 去掉页眉的横线
\fancyfoot[C]{\thepage} % 页脚居中显示页码

% 设置章节标题格式
\title{
    \Huge \textbf{西安电子科技大学} \\
    \vspace{0.5cm} 
    ——\LARGE 操作系统~大作业
}
\author{}
\date{}


\begin{document} %%%%%%%%%%%%%%%%%%%%%%%%%

\AddToShipoutPicture*{
    \AtPageUpperLeft{
        \begin{tikzpicture}[remember picture, overlay]
            \node[anchor=north west, inner sep=0pt] at (1cm,-1cm) {
                \includegraphics[width=3cm]{src/xd_local.png}
            };
        \end{tikzpicture}
    }
}

\maketitle

\begin{center}
    \Huge 实验名称: ~ \textbf{容器思想及Docker使用}
    \\
    \Large ———— ~ 基于Docker的Bert落地
\end{center}

\vspace{1cm}

% 添加个人信息部分
\begin{center}
    \Large
    \begin{tabular}{ll}
        学院: & \underline{人工智能}\\
        & \\
        行政班: & \underline{2320032}\\
        & \\
        教学班: & \underline{AI202105 | 02班}\\
        & \\
        姓名: & \underline{黄煜垒, 王小宁}\\
        & \\
        学号: & \underline{23009201388, 23009200506}\\
        & \\
        日期: & \underline{2025年10月}\\
    \end{tabular}
\end{center}

\pagenumbering{roman}
\newpage %%%%%%%%%%%%%%%%%%%%%%%%%

\tableofcontents % 目录

\newpage %%%%%%%%%%%%%%%%%%%%%%%%%
\pagenumbering{arabic}

%%%%%%%%%%%%%%%%%%%%%%%%%%%%%%%%%%%%%%%%%%%%%%%%%%%%%%%%%%%%
\section{Docker介绍} % 宁来写,黄来补


\begin{figure}[H]
    \centering
    \includegraphics[width=1\textwidth]{src/docker.png}
    \caption{docker架构图}
\end{figure}


%%%%%%%%%%%%%%%%%%%%%%%%%%%%%%%%%%%%%%%%%%%%%%%%%%%%%%%%%%%%
\subsection{为什么要Docker} % 宁来写,黄来补

%%%%%%%%%%%%%%%%%%%%%%%%%%%%%%%%%%%%%%%%%%%%%%%%%%%%%%%%%%%%
\subsection{Docker基本命令} % 宁来写,黄来补

%%%%%%%%%%%%%%%%%%%%%%%%%%%%%%%%%%%%%%%%%%%%%%%%%%%%%%%%%%%%
\section{Bert介绍} % 宁来写,黄来补

%%%%%%%%%%%%%%%%%%%%%%%%%%%%%%%%%%%%%%%%%%%%%%%%%%%%%%%%%%%%
\subsection{简述Transformer} % 宁来写,黄来补

%%%%%%%%%%%%%%%%%%%%%%%%%%%%%%%%%%%%%%%%%%%%%%%%%%%%%%%%%%%%
\subsection{简述Bert} % 宁来写,黄来补

%%%%%%%%%%%%%%%%%%%%%%%%%%%%%%%%%%%%%%%%%%%%%%%%%%%%%%%%%%%%
\section{实验环境搭建} 

%%%%%%%%%%%%%%%%%%%%%%%%%%%%%%%%%%%%%%%%%%%%%%%%%%%%%%%%%%%%
\subsection{设备介绍}

服务器配置如下:

\begin{itemize}
    \item \textbf{操作系统:} Ubuntu 20.04.6 LTS
        \footnotemark[1]
    \item \textbf{CPU:} x86\_64
        \footnotemark[2]
    \item \textbf{GPU:} 两块 NVIDIA 显卡
\end{itemize}
\footnotetext[1]{teLTS (long time support): 长期支持版xt}
\footnotetext[2]{x86\_64: 64位架构}

GPU配置如下:

\begin{itemize}
    \item \textbf{CUDA版本:} 12.4 
    \item \textbf{显卡型号:} NVIDIA GeForce RTX 4090 
    \item \textbf{单卡显存:} 24GB
\end{itemize}

\begin{figure}[H]
    \centering
    \includegraphics[width=1\textwidth]{src/nv.png}
    \caption{nvidia-smi}
\end{figure}

%%%%%%%%%%%%%%%%%%%%%%%%%%%%%%%%%%%%%%%%%%%%%%%%%%%%%%%%%%%%
\subsection{构建 Docker}

在 \texttt{CentOS} 系统中,我们通常使用 \texttt{yum} 安装 Docker;
而在 \texttt{Ubuntu} 系统中,则使用 \texttt{apt} 进行安装。
下面给出\textbf{基于 Ubuntu 的 Docker 安装步骤}:

\begin{tcolorbox}[colback=gray!5!white, colframe=gray!40!black, title=\textbf{Ubuntu 系统中 Docker 安装步骤}, breakable]
\begin{enumerate}[label={\textbf{步骤 \arabic*:}}, leftmargin=1.6cm, itemsep=0.6em]
    \item 更新本地软件包索引:
\begin{minted}[fontsize=\small, breaklines]{bash}
sudo apt-get update
\end{minted}

    \item 安装必要的依赖包:
\begin{minted}[fontsize=\small, breaklines]{bash}
sudo apt-get -y install apt-transport-https ca-certificates curl software-properties-common
\end{minted}

    \item 从阿里云镜像站下载 Docker 的 GPG 密钥:
\begin{minted}[fontsize=\small, breaklines]{bash}
curl -fsSL https://mirrors.aliyun.com/docker-ce/linux/ubuntu/gpg | sudo apt-key add -
\end{minted}

    \item 添加 Docker 稳定版仓库:
\begin{minted}[fontsize=\small, breaklines]{bash}
sudo add-apt-repository \
"deb [arch=amd64] https://mirrors.aliyun.com/docker-ce/linux/ubuntu $(lsb_release -cs) stable"
\end{minted}

    \item 更新软件包索引并安装 Docker:
\begin{minted}[fontsize=\small, breaklines]{bash}
sudo apt-get update
sudo apt-get install docker-ce docker-ce-cli containerd.io
\end{minted}

    \item 验证安装:
\begin{minted}[fontsize=\small, breaklines]{bash}
docker -v
\end{minted}
\end{enumerate}
\end{tcolorbox}

为了\textbf{让主机的 GPU 资源能够被 Docker 容器访问},
需要安装 NVIDIA 的 GPU 支持工具包 \texttt{nvidia-container-toolkit}。

\begin{tcolorbox}[colback=gray!5!white, colframe=gray!40!black, title=\textbf{NVIDIA GPU 支持配置步骤}, breakable]
\begin{enumerate}[label={\textbf{步骤 \arabic*:}}, leftmargin=1.6cm, itemsep=0.6em]
    \item 下载 NVIDIA GPG 公钥:
\begin{minted}[fontsize=\small, breaklines]{bash}
distribution=$(. /etc/os-release;echo $ID$VERSION_ID)
curl -fsSL https://nvidia.github.io/libnvidia-container/gpgkey \
| sudo gpg --dearmor -o /usr/share/keyrings/nvidia-container-toolkit-keyring.gpg
\end{minted}

    \item 添加 NVIDIA 软件仓库:
\begin{minted}[fontsize=\small, breaklines]{bash}
curl -s -L https://nvidia.github.io/libnvidia-container/$distribution/libnvidia-container.list \
| sed 's#deb https://#deb [signed-by=/usr/share/keyrings/nvidia-container-toolkit-keyring.gpg] https://#g' \
| sudo tee /etc/apt/sources.list.d/nvidia-container-toolkit.list
\end{minted}

    \item 更新列表并安装 Toolkit:
\begin{minted}[fontsize=\small, breaklines]{bash}
sudo apt update
sudo apt install -y nvidia-container-toolkit
\end{minted}

    \item 启用 GPU Runtime 并重启服务:
\begin{minted}[fontsize=\small, breaklines]{bash}
sudo nvidia-ctk runtime configure --runtime=docker
sudo systemctl restart docker
\end{minted}
\end{enumerate}
\end{tcolorbox}


接下来可以构建 Docker 镜像并创建容器了:



%%%%%%%%%%%%%%%%%%%%%%%%%%%%%%%%%%%%%%%%%%%%%%%%%%%%%%%%%%%%
\section{实验步骤}



%%%%%%%%%%%%%%%%%%%%%%%%%%%%%%%%%%%%%%%%%%%%%%%%%%%%%%%%%%%%
\subsection{启动容器(cuda)}

%%%%%%%%%%%%%%%%%%%%%%%%%%%%%%%%%%%%%%%%%%%%%%%%%%%%%%%%%%%%
\subsection{启动前端}

%%%%%%%%%%%%%%%%%%%%%%%%%%%%%%%%%%%%%%%%%%%%%%%%%%%%%%%%%%%%
\subsection{启动后端}

%%%%%%%%%%%%%%%%%%%%%%%%%%%%%%%%%%%%%%%%%%%%%%%%%%%%%%%%%%%%
\section{实机展示}

%%%%%%%%%%%%%%%%%%%%%%%%%%%%%%%%%%%%%%%%%%%%%%%%%%%%%%%%%%%%
\section{总结}


\end{document} %%%%%%%%%%%%%%%%%%%%%%%%%